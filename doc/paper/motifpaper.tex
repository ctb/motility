\documentclass{bioinfo}
\copyrightyear{2012}
\pubyear{2012}

\begin{document}
\firstpage{1}

\title[short Title]{Fast motif searching with the motility software library}
\author[Sample \textit{et~al}]{Philip Trosko\,$^{1}$, 
Eric Macdonald\,$^{2}$ 
and C. Titus Brown\,$^{3,2}$\footnote{to whom correspondence should be addressed}}
\address{$^{1}$Trosko's Consulting and Programming, 
1630 Sylvan Glen, Okemos, MI, 48864\\
$^{2}$Computer Science and Engineering, Michigan State University, 48823.\\
$^{3}$Microbiology and Molecular Genetics, Michigan State University, 48823.}

\history{Received on XXXXX; revised on XXXXX; accepted on XXXXX}

\editor{Associate Editor: XXXXXXX}

\maketitle

\begin{abstract}

\section{Motivation:}

Searching for transcription factor binding sites in DNA is now
routinely done on genomic scales.  We describe the motility library
for fast exact and inexact string searching, and compare its performance
to the performance of BioPython and TAMO motif searching.

\section{Results:}

All three software packages produced consistent results, but motility is
consistently much faster than BioPython and TAMO.  This is because
motility is primarily implemented in C++, while BioPython and TAMO
are in pure Python.

\section{Availability:}
The three routines are available online, 
and can be downloaded and compiled by a person with some sophistication 
using Git, ftp, make and python.
\section{Contact:} \href{brown@msu.edu}{brown@msu.edu}
\end{abstract}

\section{Introduction}
The field of string searching usually falls in the discipline of 
Computer Science.  Usually we refer to a query q of length m and all the
offsets of q that lie on a text of length n, where n is usually 
much longer than m.
% Figure~\ref{eq:01} follows.
Figure~\ref{eq:01} follows.

\begin{equation}
 q[m] \rightarrow t[n],   O(m*n)\label{eq:01}
\end{equation}


% TODO: Either embed caption in a fake figure with equation in math mode.
%	Or else don't use captions with equations.
% Equation~(\ref{eq:01})
%\caption{A query of length m querried against a text of length n.}. \citealp{Knuth65} 

Figure~\ref{eq:02} calculation and runtime of Position Weight String.
\begin{equation}
  \sum_{i=1}^{n} \sum_{j=1}^{m} W_i (q[j], t[i+j]),   O(m*n)\label{eq:02}
\end{equation}


% Equation~(\ref{eq:02}) \caption{The weight W of an offset i in t between q and t }.  \citealp{Sanko65} 


\section{Approach}
Our approach was to dowload Biopython, Motility and Tamo all on to 
the same machine, and run them using the same queries and the same domain data.
Once running we furthermore for reproducibility made our data run in scripts
and ran these scripts under similar conditions on the Amazon cluster.

\begin{methods}
\section{Methods}
Our method involved dividing the Motif queries into three groups:, 
\begin{itemize}
\item Exact matching,
\item IUPAC matching,
\item Positional weight matrix matching.  
\end{itemize}
All three routines support these matches,
the queries and data were from the bacteria data set. \citealp{BactXX} 
We feel they are representative of the kinds of queries of strings into
genomic information. First the exact queries were broken down by size into
four selected five groups:  TATAA, Small, Medium, Large and Very Large 
of randomly selected query data. The reason for an additional 
small recognized region called the TATAA box was because
in our test data, Bacteria, it 
has a large number of recognition sites and is a functional recognition site.  
Furthermore forward match, reverse match 
of each of  the queries was performed because in Biology 
DNA is two stranded with
matches possible in either direction.

\begin{table}[!t]
\processtable{Forward and Reverse Exact Match Timings in ecoli\label{Tab:01}}
{\begin{tabular}{lllll}\toprule
Exact Matches in ecoli\\\midrule
Query Length 	 & Matches & Biopython & Motility & TAMO \\
TATAA (5 bases)  & 5811 &52.0 & 0.01 & 44.5\\
Small (8 bases)  & 245 &52.9 & 0.01 & 71.3\\
Medium (12 bases)& 3 &52.3 & 0.01 & 98.3\\
Large (20 bases) & 1 &52.7 & 0.01 & 135.6\\
VLarge(60 bases) & 1 &52.2 & 0.01 & 378.2\\\botrule
\end{tabular}}{The exact timings were run on a Macintosh computer}
\end{table}

\begin{table}[!t]
\processtable{Forward and Reverse IUPAC Ambiguity and Position Weight Matrix 
Matches in ecoli\label{Tab:02}}
{\begin{tabular}{lll}\toprule
Routine 	& IUPAC (Matches=10099) & PWM (Matches=8505)\\
Biopython 	& 95.6 	& 77.9\\
Motility 	& 3.6 	& 3.5\\
TAMO 		& 28.0	& 151.8\\\botrule
\end{tabular}}{The IUPAC string match on string "WGTATA" and Position 
Weight Matrix on string "WGATAR" with a threshhold of 10.57.
Timings were run on a Macintosh computer}
\end{table}

\end{methods}

\section{Discussion}
We feel the text and the size of data and queries are very representative 
of the kinds of ongoing queries in biology. When trying to find
epigenetic structure in a single small organism's DNA,
small restriction site cuts or operative regions in the DNA all 
three routines are adequate.
If doing a simple exact match a hard coded string search would be fast,
however Motility is fastest.  
Biopython and Tamo are slower.


\section{Conclusion}
Our brief software run time comparison shows Motility 
to be superior in run times and that all software compared
are capable of doing the searches required for simple DNA comparison.

\section*{Acknowledgement}
We would like recognise??

\paragraph{Funding\textcolon} 
Titus Brown was provided for by XXXXXX, 
and funding for Eric Macdonald was provided by XXXXXX.

%\bibliographystyle{natbib}
%\bibliographystyle{achemnat}
%\bibliographystyle{plainnat}
%\bibliographystyle{abbrv}
%\bibliographystyle{bioinformatics}
%
%\bibliographystyle{plain}
%
%\bibliography{Document}


\begin{thebibliography}{}

\bibitem[Knuth {\it et~al}., 1965]{Knuth65} 
Knuth,D.(1965) Volume 3, Chapter 6, Searching 
\textit {The Art of Computer Programming}, pp-pp.

\bibitem[Sankoff and Kruskal., 1965]{Sanko65} 
Sankoff, X and Kruskal,X.(XXXX), Chapter X, SXXXXXX 
\textit {Time Warps, String Edits and Macromolecules, The Art and Practice of
String Comparison}, pp-pp.

\bibitem[Escherichia.,xxxx] {BactXX} Name, X. (xxxx) Escherichia coli, compete genome. 
% >mg1655 |ref|NC_000913.2| 
%Escherichia coli str. K-12 substr. MG1655 chromosome, complete genome

\end{thebibliography}
\end{document}
